\section{Conclusions}






\subsection{}
%%%%%%%%%%%%%%%%%%%%%%%%%%%%%%%%%%%%%%%%%%%%%%%%%%%%%%%%
{
\paper{Zia et al., 2017 in {\bf Computer Assisted Radiology and Surgery};
Mori 2012 in {\bf Development and Learning and Epigenetic Robotics};
Mitsukura et al., 2017 in {\bf Electroencephalography}; 
Marwan et al. 2019 in {\bf http://recurrence-plot.tk/}
}

\begin{frame}{Applications of Nonlinear Dynamics}
    \begin{figure}
        %\centering
        \includegraphicscopyright[width=0.99\linewidth]
	{applications/drawing}{}
	
	%\caption{(A) Normalised, (B) \textt{sgolay(p=5,n=25)}, and (C) \textt{sgolay(p=5,n=159)} } 
   \end{figure}
	
\end{frame}
}








\subsection{}
%%%%%%%%%%%%%%%%%%%%%%%%%%%%%%%%%%%%%%%%%%%%%%%%%%%%%%%%
{
%\paper{}

\begin{frame}{OA Publications}

\tiny
 
\textbf{PEER-REVIEW CONFERENCE PAPERS}
\begin{itemize}	
	\item \textit{Towards the Analysis of Movement Variability in Human-Humanoid Imitation Activities} 
	(HAI2017) 
	\item \textit{Towards the Quantification of Human-Robot Imitation Using Wearable Inertial Sensors} (HRI2017)
	\item \textit{Analysis of the Movement Variability in Dance Activities using Wearable Sensors} (WeRob2016)
	\item \textit{Understanding Movement Variability of Simplistic Gestures Using an Inertial Sensor} (PerDis2016)
\end{itemize}

\textbf{PREPRINTS \& in preparation}
\begin{itemize}	
	\item \textit{Strengths and weaknesses of Recurrence Quantification Analysis in the context of human-humanoid interaction}
	(ArXiv, October 2018) for Scientific Reports.
	\item \textit{3D surface plots of RQA Shannon Entropy} \\
 	for Frontiers in Applied Mathematics and Statistics.
\end{itemize}

\textbf{TALKS}
\begin{itemize}	
	\item \textit{Quantifying the Inherent Chaos of Human Movement Variability} \\
	15th Experimental Chaos and Complexity Conference 
	\item \textit{Towards the Analysis of Movement Variability for Facial Expressions with
	Nonlinear Dynamics} \\
	The 7th Consortium of European Research on Emotion Conference 
\end{itemize}

	
\end{frame}
}




\subsection{}
%%%%%%%%%%%%%%%%%%%%%%%%%%%%%%%%%%%%%%%%%%%%%%%%%%%%%%%%
{
\paper{Xochicale 2019 in {\bf PhD thesis}}

\begin{frame}{FIRST Open Access PhD Thesis at UoB (since 1900)}
    \begin{figure}
        %\centering
        \includegraphicscopyright[width=0.99\linewidth]
	{oathesis/drawing}{}	
	%\caption{ } 
   \end{figure}
	
\end{frame}
}



