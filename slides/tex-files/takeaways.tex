\section{Takeaways}



\subsection{}
%%%%%%%%%%%%%%%%%%%%%%%%%%%%%%%%%%%%%%%%%%%%%%%%%%%%%%%%
{
\paper{Peng 2011 in {\bf Science} DOI: 10.1126/science.1213847}

\begin{frame}{My journey in OA according to the Reproducibility Spectrum}

    \begin{figure}
        %\centering
        \includegraphicscopyright[width=0.95\linewidth]
	{peng2011/drawing}{}
	%\caption{(A) Normalised, (B) \textt{sgolay(p=5,n=25)}, and (C) \textt{sgolay(p=5,n=159)} } 
   \end{figure}
	
\end{frame}
}





\subsection{}
%%%%%%%%%%%%%%%%%%%%%%%%%%%%%%%%%%%%%%%%%%%%%%%%%%%%%%%%
{
%\paper{Ainsworth 2019, https://doi.org/10.6084/m9.figshare.11762121}
{
\paper{Open Science Logo (2020), https://github.com/jcolomb/opensciencelogo}

\begin{frame}{Takeaways}


\begin{columns}



\begin{column}{0.6\textwidth}


 
\begin{enumerate}
\item Learn from other open access scientists and open source enthusiasts 
	(there are quite amazing human beings in \faTwitter and \faGithub)
\item 	Make use of free and open source 
	software and hardware as much as you can, and
	 perhaps also contribute to it. (happy to help!)
\item Read, replicate, learn, fail, share 
	and do not stop exploring
	the exciting world of open access science!  
\end{enumerate}
\end{column}




\begin{column}{0.4\textwidth} 
    \begin{figure}
        %\centering
        \includegraphicscopyright[width=0.8\linewidth]
	{open-science-logo-flower/drawing}{}	
	%\caption{ } 
   \end{figure}
	
\end{column}



\end{columns}




\end{frame}
}



