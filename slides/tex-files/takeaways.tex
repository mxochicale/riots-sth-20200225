\section{Takeaways}



\subsection{}
%%%%%%%%%%%%%%%%%%%%%%%%%%%%%%%%%%%%%%%%%%%%%%%%%%%%%%%%
{
\paper{Peng 2011 in {\bf Science} DOI: 10.1126/science.1213847}

\begin{frame}{My journey in OA according to the Reproducibility Spectrum}

    \begin{figure}
        %\centering
        \includegraphicscopyright[width=0.99\linewidth]
	{peng2011/drawing}{}
	%\caption{(A) Normalised, (B) \textt{sgolay(p=5,n=25)}, and (C) \textt{sgolay(p=5,n=159)} } 
   \end{figure}
	
\end{frame}
}





\subsection{}
%%%%%%%%%%%%%%%%%%%%%%%%%%%%%%%%%%%%%%%%%%%%%%%%%%%%%%%%
{
%\paper{Ainsworth 2019, https://doi.org/10.6084/m9.figshare.11762121}
{
\paper{Gema Bueno de la Fuente  (2019), https://www.fosteropenscience.eu/node/1420}

\begin{frame}{Takeaways}
 
\begin{enumerate}
\item Learn from other open access scientists and open source enthusiasts 
	(there are quite amazing human beings in Twitter and GitHub)
\item Use, perhaps also contribute to, free and open source 
	software as much as you can (happy to help!)
\item How to start?: Reproduce a paper in ReScience
\item Read, replicate, learn, fail, share 
	and do not stop exploring
	the exciting world of open access science!  
\end{enumerate}

	
    \begin{figure}
        %\centering
        \includegraphicscopyright[width=0.5\linewidth]
	{conclusions/src/OpenScienceBuildingBlocks}{}	
	%\caption{ } 
   \end{figure}
	


\end{frame}
}



%
%
%\subsection{}
%%%%%%%%%%%%%%%%%%%%%%%%%%%%%%%%%%%%%%%%%%%%%%%%%%%%%%%%%
%{
%\paper{Gema Bueno de la Fuente  (2019), https://www.fosteropenscience.eu/node/1420}
%
%\begin{frame}{}
% 
%\large 
%Pushing the frontiers of knowledge is indeed a great challenge 
%but no less important than making such knowledge 
%open accessible and reproducible.	
%
%    \begin{figure}
%        %\centering
%        \includegraphicscopyright[width=0.6\linewidth]
%	{conclusions/src/OpenScienceBuildingBlocks}{}	
%	%\caption{ } 
%   \end{figure}
%	
%
%
%\end{frame}
%}
%
%
%




% https://www.researchgate.net/publication/337935267_Open_science_Leaving_no_one_behind


