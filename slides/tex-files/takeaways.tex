\section{Takeaways}


\subsection{}
%%%%%%%%%%%%%%%%%%%%%%%%%%%%%%%%%%%%%%%%%%%%%%%%%%%%%%%%
{
\paper{Ainsworth 2019, https://doi.org/10.6084/m9.figshare.11762121}

\begin{frame}{Takeaways}
 
\begin{enumerate}
\item Make your work available in a stable, version-controlled repository
\item Choose an open source license to allow adoption and reuse
\item Include instructions and examples of how to cite your software in its
\end{enumerate}

	
\end{frame}
}





\subsection{}
%%%%%%%%%%%%%%%%%%%%%%%%%%%%%%%%%%%%%%%%%%%%%%%%%%%%%%%%
{
\paper{Gema Bueno de la Fuente  (2019), https://www.fosteropenscience.eu/node/1420}

\begin{frame}{}
 
\large 
Pushing the frontiers of knowledge is indeed a great challenge 
but no less important than making such knowledge 
open accessible and reproducible.	

    \begin{figure}
        %\centering
        \includegraphicscopyright[width=0.6\linewidth]
	{conclusions/src/OpenScienceBuildingBlocks}{}	
	%\caption{ } 
   \end{figure}
	


\end{frame}
}




\subsection{}
%%%%%%%%%%%%%%%%%%%%%%%%%%%%%%%%%%%%%%%%%%%%%%%%%%%%%%%%
{
\paper{Peng 2011 in {\bf Science} DOI: 10.1126/science.1213847}

\begin{frame}{Reproducibility Spectrum}
    \begin{figure}
        %\centering
        \includegraphicscopyright[width=0.99\linewidth]
	{peng2011/src/fig01}{}
	
	%\caption{(A) Normalised, (B) \textt{sgolay(p=5,n=25)}, and (C) \textt{sgolay(p=5,n=159)} } 
   \end{figure}
	
\end{frame}
}






% https://www.researchgate.net/publication/337935267_Open_science_Leaving_no_one_behind


