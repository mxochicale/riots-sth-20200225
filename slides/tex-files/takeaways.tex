\section{Takeaways}



\subsection{}
%%%%%%%%%%%%%%%%%%%%%%%%%%%%%%%%%%%%%%%%%%%%%%%%%%%%%%%%
{
\paper{Peng 2011 in {\bf Science} DOI: 10.1126/science.1213847}

\begin{frame}{My journey in OA according to the Reproducibility Spectrum}

    \begin{figure}
        %\centering
        \includegraphicscopyright[width=0.99\linewidth]
	{peng2011/drawing}{}
	%\caption{(A) Normalised, (B) \textt{sgolay(p=5,n=25)}, and (C) \textt{sgolay(p=5,n=159)} } 
   \end{figure}
	
\end{frame}
}





\subsection{}
%%%%%%%%%%%%%%%%%%%%%%%%%%%%%%%%%%%%%%%%%%%%%%%%%%%%%%%%
{
%\paper{Ainsworth 2019, https://doi.org/10.6084/m9.figshare.11762121}
{
\paper{Gema Bueno de la Fuente  (2019), https://www.fosteropenscience.eu/node/1420}

\begin{frame}{Takeaways}
 
\begin{enumerate}
\item How to start?: Reproduce a paper following ReScience
\item Use free and open source software as much as you can
	(happy to help!)
\item Follow, read, replicate, learn, fail, share 
	and never stop exploring
	the exciting world of open access science!  
\end{enumerate}

	
    \begin{figure}
        %\centering
        \includegraphicscopyright[width=0.6\linewidth]
	{conclusions/src/OpenScienceBuildingBlocks}{}	
	%\caption{ } 
   \end{figure}
	


\end{frame}
}



%
%
%\subsection{}
%%%%%%%%%%%%%%%%%%%%%%%%%%%%%%%%%%%%%%%%%%%%%%%%%%%%%%%%%
%{
%\paper{Gema Bueno de la Fuente  (2019), https://www.fosteropenscience.eu/node/1420}
%
%\begin{frame}{}
% 
%\large 
%Pushing the frontiers of knowledge is indeed a great challenge 
%but no less important than making such knowledge 
%open accessible and reproducible.	
%
%    \begin{figure}
%        %\centering
%        \includegraphicscopyright[width=0.6\linewidth]
%	{conclusions/src/OpenScienceBuildingBlocks}{}	
%	%\caption{ } 
%   \end{figure}
%	
%
%
%\end{frame}
%}
%
%
%




% https://www.researchgate.net/publication/337935267_Open_science_Leaving_no_one_behind


