\section{Tools}


\subsection{}
%%%%%%%%%%%%%%%%%%%%%%%%%%%%%%%%%%%%%%%%%%%%%%%%%%%%%%%%
{
\paper{https://www.gnu.org/philosophy/free-sw.html}

\begin{frame}{Free Software} 

“Free software” means that the users have the freedom 
to run, copy, distribute, study, change and improve the software

    \begin{figure}
        %\centering
        \includegraphicscopyright[width=0.95\linewidth]
	{foss/src/charac-more}{}
	%\caption{ } 
   \end{figure}
	


\end{frame}
}


\subsection{}
%%%%%%%%%%%%%%%%%%%%%%%%%%%%%%%%%%%%%%%%%%%%%%%%%%%%%%%%
{
\paper{https://opensource.org/about}

\begin{frame}{Open Source Software} 

%"enables a development method for software that harnesses the power of distributed peer review and transparency of process. 
"The promise of open source is higher quality, better reliability, greater flexibility, lower cost, and an end to predatory vendor lock-in."


    \begin{figure}
        %\centering
        \includegraphicscopyright[width=0.7\linewidth]
	{foss/src/foss_logos}{}
	%\caption{ } 
   \end{figure}
	


\end{frame}
}



\subsection{}
%%%%%%%%%%%%%%%%%%%%%%%%%%%%%%%%%%%%%%%%%%%%%%%%%%%%%%%%
{
%\paper{https://opensource.org/about}

\begin{frame}{My Collection of Scientific Tools} 

    \begin{figure}
        %\centering
        \includegraphicscopyright[width=0.95\linewidth]
	{mytools/drawing.pdf}{}
	%\caption{ } 
   \end{figure}
	


\end{frame}
}





\subsection{}
%%%%%%%%%%%%%%%%%%%%%%%%%%%%%%%%%%%%%%%%%%%%%%%%%%%%%%%%
{
\paper{Edsger W.Dijkstra. 1975. How do we tell truths that might hurt?}

\begin{frame}{}
 
\Large 
\textit{ 
The tools we use have profound (and devious!) influence on our 
thinking habits, and therefore, on our thinking abilities.}

\end{frame}
}




